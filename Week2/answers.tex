\documentclass{exam}
\usepackage{xcolor, minted, graphicx, fontspec}
\setmainfont{Open Sans}
\graphicspath{ {./images} }

\author{Jordy Alkema}
\title {Homework Week 2}

\begin{document}
\maketitle

\section{Review}
\begin{questions}
	\question
	\begin{parts}
		\part
		5, 1 voor elk van de volgende tables: test, unit of measure, variable, sample, organization
		\part
		Ik zou de tabel opsplitsen in een overkoepelende tabel voor observation met: date, time en value en twee tabellen voor physical observations en test observations, om zo op de volgende structuur uit te komen.

		\textbf{observation:}
		\begin{itemize}
			\item id (PK)
			\item unit (FK)
			\item variable (FK)
			\item party (FK)
			\item date
			\item time
			\item value
		\end{itemize}
		\textbf{test\_observation}
		\begin{itemize}
			\item id (PK)
			\item observation (FK)
			\item test (FK)
		\end{itemize}
		\textbf{physical\_observation}
		\begin{itemize}
			\item id (PK)
			\item observation (FK)
			\item sample (FK)
		\end{itemize}
		\part
		De volgende waarden zijn een record voor een physical observation

		\begin{center}
			\textbf{observation:}

			\begin{tabular}{ | c | c | c | c | c | c | c |}
				\hline
				id & unit & variable & party & date       & time  & value \\
				\hline
				1  & 1    & 1        & 1     & 08-03-2023 & 12:00 & 7     \\
				\hline
			\end{tabular}

			\textbf{physical\_observation}

			\begin{tabular}{ | c | c | c | }
				\hline
				id & observation & sample \\
				\hline
				1  & 1           & 1      \\
				\hline
			\end{tabular}
		\end{center}
	\end{parts}
	\pagebreak
\end{questions}

\section{Questions}
\begin{questions}
	\setcounter{question}{1}
	\question
	\begin{parts}
		\part
		\includegraphics[width=\textwidth,height=\textheight,keepaspectratio]{question2a}
		\part
		\includegraphics[]{question2b}

		Door de fields nullbable te maja ken en eventueel constraints toe te passen kan ervoor worden gezorgd dat alle typen posts in 1 table kunnen worden opgeslagen
		\part
		Ze zijn gecompliceerder dan traditionele SQL relations. De queries die geschreven moeten worden/de orm methods die gebruikt moeten worden zijn gecompliceerder en slechter overzichtelijk dan normaal gesproken. Hierdoor wordt de database minder overzichtelijk en in de toekomst mogelijk lastiger te onderhouden.
	\end{parts}
	\question
	\includegraphics[width=\textwidth,height=\textheight,keepaspectratio]{question3}
	\question
	\includegraphics[width=\textwidth,height=\textheight,keepaspectratio]{question4}

	Een customer heeft 0 of meer deliveries, deze deliveries hebben allemaal 1 drone, 1 drone kan echter 0 of meer deliveries hebben.
	Een drone heeft 1 parent company en 1 company kan 1 of meer drones hebben. Elke delivery heeft 1 record voor weather conditions, route informatie(tijd die route heeft geduurd etc.). Elke delivery heeft 1 of meer packages en twee locations(source, destination). Verder kan een delivery status updates hebben (bijvoorbeeld huidige locatie) die op worden geslagen in flight.

	\pagebreak
	\question
	\begin{parts}
		\part
		\includegraphics[width=\textwidth,height=\textheight,keepaspectratio]{question5a}
		Een student kan zich inschrijven voor meerdere thema's in de koppeltabel student\_theme. Een thema kan aan meerdere modules worden gekoppeld en vice versa door een koppeltabel. Aan elke module kunnen meerdere toetsen worden gekoppeld. Bij elke module kan een cijfer horen dat gekoppeld wordt aan een specifieke student.
		\part
		Zie hierboven
		\part
		\includegraphics[width=\textwidth,height=\textheight,keepaspectratio]{question5c}
		\part
		\includegraphics[width=\textwidth,height=\textheight,keepaspectratio]{question5d}
		\part
		\includegraphics[width=\textwidth,height=\textheight,keepaspectratio]{question5e}
		\part
		\includegraphics[width=\textwidth,height=\textheight,keepaspectratio]{question5f}
		\part
		\includegraphics[width=\textwidth,height=\textheight,keepaspectratio]{question5g}

	\end{parts}
	\question
	\includegraphics[width=\textwidth,height=\textheight,keepaspectratio]{question6}
\end{questions}

\end{document}
